%! Author = joels
%! Date = 14/06/2021

\section{Thread synchronisation}
\textbf{Threads teilen sich Adressraum und Heap.}
\subsection{Race Condition}
Wenn mehrere Threads das Selbe Objekt lesen und anhand vom Resultat dieses überschreiben. \textcolor{b}{Time of Check and Time of Use Problem: Operationen sind nicht genügend Atomar: Ein andere Thread könnte dazwischen schreiben. $\rightarrow$ Visibility oder Lost Update problem}
\subsection{Synchronized (Monitor Lock)}
\textcolor{red}{Überholproblem und keine garantierte Reihenfolge}\\
Das Keyword \textcolor{b}{synchronized} belegt einen Lock für das Objekt.\\
$\rightarrow$ Nur ein Thread kann eine synchronized Methode in derselben Instanz zur gleichen Zeit ausführen.
\textcolor{b}{wait()} gibt lock frei und wartet auf Signal. \textcolor{b}{notify()} signalisiert nur einen beliebigen (zufälligen), wartenden Thread. $\rightarrow$ Wartet vlt auf andere Bedingung und kann zu ewigem Warten führen\\
\textcolor{b}{notifyAll()} signalisiert alle wartenden Threads.\\
\textbf{Wichtig:}
\begin{itemize}[topsep=0pt, leftmargin=3mm]
    \setlength\itemsep{-0.3em}
    \item Signalisierender Thread behält Monitor
    \item Thread ist im inneren Warteraum, bis Signal kommt
    \item Bei notify() kommt Thread wieder in äusseren Warteraum
\end{itemize}
\begin{lstlisting}
class BoundedBuffer <T> {
  private Queue<T> queue = new LinkedList<>();
  private int limit = 1; // or initialize in constructor
  public synchronized void put(T item) throws InterruptedException {
    while (queue.size () == limit) {
      wait(); // await non full
    } queue.add(item); notifyAll(); } // signal non empty
  public synchronized T get() throws InterruptedException {
    while (queue.size () == 0) {
      wait();// await non empty
    } var item = queue.remove(); notifyAll(); // non full
    return item; } }
// synchronized(object){ statements } ist eine explizite Angabe, auf welcher Instanz gelockt wird:
public void deposit(int amount) {
    synchronized(this) {this.balance += amount;} }
\end{lstlisting}