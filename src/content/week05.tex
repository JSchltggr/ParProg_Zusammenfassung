%! Author = joels
%! Date = 14/06/2021

\section{Thread Pools}
\subsection{Konzept und Funktionsweise}
\textcolor{b}{\textbf{Tasks:}} Implementieren potentiell paralelle Arbeitspakete. Auszuführende Tasks werden in Warteschlange eingereiht.\\
\textcolor{b}{\textbf{Thread Pool:}} Beschränke Anzahl von Worker-Threads. Holen Tasks aus der Queue und führen sie aus.
\subsection{Vorteile und Einschränkungen}
\textcolor{b}{\textbf{Vorteile:}}
\begin{itemize}[topsep=0pt, leftmargin=3mm]
    \setlength\itemsep{-0.3em}
    \item Beschränkte Threadanzahl (Zu viele verlangsamen System)
    \item Recycling der Threads (Thread-Erzeugung und Freigabe)
    \item Höhere Abstraktion (Trenne TaskBeschreibung/Ausführung)
    \item Anzahl Threads pro System konfigurierbar
\end{itemize}
\textcolor{b}{\textbf{Einschränkungen:}}
\begin{itemize}[topsep=0pt, leftmargin=3mm]
    \setlength\itemsep{-0.3em}
    \item Tasks dürfen nicht aufeinander warten (Deadlock)
    \item Task muss zu Ende laufen, bevor Worker Thread anderen Task ausführen kann (Ausnahme: Geschachtelte Tasks)
\end{itemize}
\subsection{Fork \& Join Pool}
\textcolor{b}{\textbf{Future Konzept:}} Repräsentiert ein zukünftiges Resultat. Proxy wartet auf Resultat, muss das Ende der Berechnung abwarten.
\begin{lstlisting}
// Future Verwendung
var threadPool = new ForkJoinPool();
Future<Integer> future = threadPool.submit(() -> {
    return value; }); // Tasks können Rückgabe haben
int result = future.get() // Blockiert, bis Task beendet
// Einfaches Beispiel:
var left = threadPool.submit(() -> count(leftPart));
var right = threadPool.submit(() -> count(rightPart));
result = left.get() + right.get();
\end{lstlisting}
\subsubsection{Rekursive Tasks}
Tasks können Untertasks starten und abwarten. Erben von RecursiveTask$<$T$>$
\begin{itemize}[topsep=0pt, leftmargin=3mm]
    \setlength\itemsep{-0.3em}
    \item T compute(): Task Implementierung
    \item fork(): Starte als Sub-Task in einem anderen Task
    \item T join(): Warte auf Task-Ende und frage Resultat ab
    \item T invoke(): Ein Sub-Task starten und abwarten
    \item invokeAll(): Mehrere Sub-Tasks starten und abwarten
\end{itemize}
\begin{lstlisting}
class ConvolutionTask extends RecursiveAction {
  private final int[] input;
  private final int[] output;
  private final int lower;
  private final int upper;
  private final static int THRESHOLD = 1000;
  public ConvolutionTask(int[] input, int[] output, int lower, int upper) {
    this.input = input; this.output = output;
    this.lower = lower; this.upper = upper; }
  @Override protected void compute() {
    if (upper - lower > THRESHOLD) {
      int middle = (lower + upper) / 2;
    invokeAll(
      new ConvolutionTask(input, output, lower,middle),
      new ConvolutionTask(input, output, middle, upper));
    } else { for (int i = lower; i < upper; i++) {
        int left = i > 0 ? input[i - 1] : 0;
        int right = i < input.length - 1 ? input[i+1] : 0;
        output[i] = left + input[i] + right; } } } }
\end{lstlisting}
\subsection{Asynchrone Programmierung}
\textcolor{b}{\textbf{Ziel:}} Aufrufer soll während der Operation weiterarbeiten. $\rightarrow$ Operation in Thread oder Thread Pool auslagern
\begin{lstlisting}
// Reihenfolge der Downloads und GUI erhalten bleiben
void download(List<URL> links, OutputStream output) {
  if (links.isEmpty()) { statusLabel.setText("Done"); }
  var url = links.get(0);
  statusLabel.setText("Downloading " + url);
    if (cancelBox.isSelected()) {
      statusLabel.setText("Cancelled!");
      return; }
    CompletableFuture.runAsync(() -> {
      url.openStream().transferTo(output);
      var remaining = links.subList(1, links.size());
      SwingUtilities.invokeLater(() -> download(remaining, output)); }); }
// Asynchron mit CompletableFuture
CompletableFuture<File> zipAsync(File[] files) {
  statusLabel.setText("zip started");
  return CompletableFuture.supplyAsync(() -> {
    File[] temp = new File[files.length];
    for (int i = 0; i < files.length; i++) {
      temp[i] = compress(files[i]); }
    SwingUtilities.invokeAndWait(() -> statusLabel.setText("all files compressed"));
    File output = archive(temp);
    SwingUtilities.invokeAndWait(() -> statusLabel.setText("zip completed"));
    return output; }); }
// Parallel und asynchron
void scanAsync(List<File> list, String pattern) {
  List<CompletableFuture<Void>>futures=new ArrList<>();
  for (File file : list) {
    futures.add(CompletableFuture.runAsync(() -> {
      if (search(file, pattern)) {
        print("Found " + file); } })); } //SwingUtilities mit invokeAndWait falls auf GUI schreiben
  CompletableFuture.allOf(futures.toArray(new
CompletableFuture[0])).thenRun(() -> print("Done")); }
\end{lstlisting}
\subsubsection{Continuation}
\textcolor{b}{\textbf{Ziel:}} Folgeaufgabe an asynchrone Aufgabe anhängen. Ausführung der Continuation durch beliebigen Thread.
\begin{lstlisting}
// Continuation:
future.thenAccept(result -> syso(result));
// Tasks verketten:
future.thenApplyAsync(second).thenAcceptAsync(third);
// Multi-Continuation:
CompletableFuture.allOf(future1, future2)
.thenAcceptAsync(continuation); //Warten aufeinander
CompletableFuture.any(future1, future2)
.thenAcceptAsync(continuation)//Sobald einer fertig ist
// runAsync() ist Fire and Forget. Workers sind Deamons und ignoriert Exceptions
\end{lstlisting}